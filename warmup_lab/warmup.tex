\documentclass[12pt]{article}
% \documentclass[conference]{IEEEtran}

% Packages
% Packages

% \usepackage{fancyhdr} % Required for custom headers
% \usepackage{lastpage} % Required to determine the last page for the footer
% \usepackage{extramarks} % Required for headers and footers
% \usepackage[usenames,dvipsnames]{color} % Required for custom colors
% \usepackage{graphicx} % Required to insert images
% \usepackage{listings} % Required for insertion of code
% \usepackage{courier} % Required for the courier font
% \usepackage{dsfont} % For special math characters
% \usepackage{verbatim}

%\usepackage{amsmath, amssymb, bm} % For matrix notation
\usepackage[english]{babel}
\usepackage[paperwidth=8.5in,paperheight=11in,margin=1.0in]{geometry}
\usepackage{listings}
\usepackage{hyperref}
%\usepackage[cmex10]{amsmath, bm}
\usepackage{amsmath, bm}
\usepackage{blkarray}








% formatting
\pdfcompresslevel0

\lstset{ columns=flexible, breaklines=true, basicstyle=\small\ttfamily}
\setcounter{MaxMatrixCols}{20}
\linespread{1.3} %1.5x line spacing




\begin{document}
\title{Can we NLP a title?}

\author{
Elijah Bernstein-Cooper, Ben Conrad, Ahmed Saif
}
\maketitle

% ------------------------------------------------------------------------------
\section{Introduction}
% ------------------------------------------------------------------------------

    Under the context of natural language processing, this lab explores the
    relation between job descriptions and salaries.  This topic was the focus
    of a \href{http://www.kaggle.com/c/job-salary-prediction}{Kaggle}
    competition whose sponsor, Adzuna, had a database of job listings of which
    only half provided salary information (the winner recieved \$3000).  As
    applicants will more likely apply to descriptions that give a salary,
    Adzuna"s placement rate (and hence revenue) is improved if they can provide
    an estimated salary for those descriptions that did not originally include
    one.  (The employee recruting business is structured so that Adzuna
    generally can"t directly ask the companies to provide salary estimates.)
    This is challenging from the legal standpoint, as grossly incorrect
    salaries may expose Adzuna to claims from applicants and companies, and
    applicant experience, since Adzuna"s estimates must seem plausible to
    applicants before they will be willing to spend the time applying.

    While Adzuna could manually estimate these salaries, scalability encourages
    throwing computers at the problem.  In this lab we will be using Adzuna"s
    job description and salary datasets, divided into training and test sets.
    These descriptions vary in word count, industry, employment level, and
    company location, while the salaries are the mean of the provided salary
    range.  The variability in description content leads to a notoriously
    sparse matrices, so we will be interested in the tradeoffs of various
    feature descriptors.  The naieve approach to this problem is to count the
    occurrences of individual words and associate them to salaries; here each
    word is a feature and as there are many descriptive words the resulting
    matrices will be sparse.  Other feature choices may be individual word
    length, occurrences of word pairs or triplets (ie "technical
    communication"), n-grams (sequences of n characters), and many others.
    Note that it is common to ignore stop words like "the","a","it","you","we",
    etc... because they add little information.

% ------------------------------------------------------------------------------
\section{Warm-Up}
% ------------------------------------------------------------------------------

    Our goal in the warm-up is to use two descriptions for jobs with known
    salaries to predict the salary of another job given the description. Here
    are two examples from the dataset: 
    
    \begin{lstlisting}

        Engineering Systems Analyst Dorking Surrey Salary ****K Our client is
        located in Dorking, Surrey and are looking for Engineering Systems
        Analyst our client provides specialist software development Keywords
        Mathematical Modelling, Risk Analysis, System Modelling, Optimisation,
        MISER, PIONEEER Engineering Systems Analyst Dorking Surrey Salary ****K

    \end{lstlisting}

    with a salary of \$25,000 and 

    \begin{lstlisting}

        A subsea engineering company is looking for an experienced Subsea Cable
        Engineer who will be responsible for providing all issues related to
        cables. They will need someone who has at least 1015 years of subsea
        cable engineering experience with significant experience within
        offshore oil and gas industries. The qualified candidate will be
        responsible for developing new modelling methods for FEA and CFD. You
        will also be providing technical leadership to all staff therefore you
        must be an expert in problem solving and risk assessments. You must
        also be proactive and must have strong interpersonal skills. You must
        be a Chartered Engineer or working towards it the qualification. The
        company offers an extremely competitive salary, health care plan,
        training, professional membership sponsorship, and relocation package
    
    \end{lstlisting} having a salary of \$85,000.

    One method we can use these descriptions to predict the salary from another
    description is by least squares. We would like to determine the words which
    best predict salary, or even better the frequency of the words which best
    predict salary. Here we show the first 11 frequencies shown alphabetically
    for each description:

    %\newline

    \begin{center}
    \begin{minipage}[t]{.4\textwidth}
    Description 1
    \newline
    \newline
        \begin{tabular}{l|c}
            ****k & 2 \\
            analysis & 1\\
            analyst & 3\\
            and & 1\\
            are & 1\\
            client & 2\\
            development & 1\\
            dorking & 3\\
            engineering & 3\\
            for & 1\\
            in & 1 \\
        \end{tabular}
    \end{minipage}
    \begin{minipage}[t]{.4\textwidth}
    Description 2
    \newline
    \newline
        \begin{tabular}{l|c}
            1015 & 1 \\
            a & 2 \\
            all & 2 \\
            also & 2\\
            an & 3 \\
            and & 5 \\
            assessments & 1 \\
            at & 1 \\
            be & 6 \\
            cable & 2 \\
            cables & 1\\
        \end{tabular}
    \end{minipage}
    \end{center}
    %\newline

    We can collect these word counts into the matrix $\bm{A}$, and the salaries
    into the vector $\bm{b}$. $\bm{A}$ will have 2 rows, one for each
    description and as many columns as there are unique words between the two
    descriptions. $\bm{b}$ will have 2 rows, one salary for each description,
    and one column. $\bm{A}$ and $\bm{b}$ will look like the following

    %\begin{centering}
    \begin{equation*}
        \bm{A} = 
        \begin{bmatrix}
            2 & 1 & 3 & 1 & 1 & 2 & 1 & 3 & 3 & 1 & 1 \cdots \\
            1 & 2 & 2 & 2 & 3 & 5 & 1 & 1 & 6 & 2 & 1 \cdots \\
        \end{bmatrix}
    \end{equation*}

    %\newline

    \begin{equation*}
        \bm{b} = 
        \begin{bmatrix}
        2500\\
        85000
        \end{bmatrix}
    \end{equation*}

    We can then set up our problem as 

    \begin{equation}\label{eq:linsolve}
        \bm{b} = \bm{Ax}
    \end{equation}

    \noindent where $\bm{x}$ contains the weights, or importance of each word
    in predicting the salary of the job. We can find a best-fit solution to
    $\bm{x}$, $\bm{\hat{x}}$ by minimizing the residuals between $\bm{b}$ and
    $\bm{Ax}$, or rather minimizing

    \begin{equation}
        \|\bm{b} - \bm{Ax}\|^2_2
    \end{equation}

    \noindent which is also known as the sum of squared residuals. This
    optimization has a well-known solution for $\bm{\hat{x}}$

    \begin{equation}
        \bm{\hat{x}} = (\bm{A}^{T}\bm{A})^{-1}\bm{A}^T\bm{b}
    \end{equation}

    Using the solution for $\bm{\hat{x}}$, our least-squares solution to this
    problem, $\bm{\hat{x}}$ is:

    %\newline

    \begin{equation*}
        \bm{\hat{x}} = 
        \begin{blockarray}{[c] l}
            1819.6517 & "be" from [0, 6] \\
            1730.9558 & "and" from [1, 5] \\
            1427.6805 & "for" from [1, 4] \\
            1250.2887 & "engineering" from [3, 2] \\
            1213.1011 & "must" from [0, 4] \\
            1213.1011 & "will" from [0, 4] \\
            1213.1011 & "you" from [0, 4] \\
            909.8258 & "an" from [0, 3] \\
            909.8258 & "subsea" from [0, 3] \\
            909.8258 & "the" from [0, 3] \\
            732.4341 & "modelling" from [2, 1] \\
        \vdots & \vdots
        \end{blockarray}
    \end{equation*}

    For two samples, it should not be surprising that the most heavily-weighted
    words are unique to each description. Next we wish to predict the salary of
    another description using our $\bm{\hat{x}}$. 

    \begin{lstlisting}
    
        Our client is part of an international hotel chain that require an
        experienced Cluster Revenue Manager to be based in Hertfordshire. The
        Cluster Revenue Manager will drive and influence revenue for three to
        four hotels. As Cluster Revenue Manager you will maximise revenue,
        market share and profits for multiple hotels through the strategic
        coordination of revenue management processes and procedures. The
        Cluster Revenue Manager will drive the continued development and growth
        of customer service standards, revenue and profits from multiple hotels
        and to deliver the company’s mission relating to profit, people,
        customer and quality. You will currently be a Cluster Revenue Manager
        or a Regional/ Area Revenue Manager looking after a minimum of two
        propertys or a Revenue Manager in a large unit managing both rooms and
        conference space. This job was originally posted as
        www.caterer.com/JobSeeking/ClusterRevenueManager_job****

    \end{lstlisting} having a salary of \$45,000.

    We now construct the matrix $\bm{A}$ for this description. This matrix will
    only contain frequencies of words that were present in the previous two
    descriptions, so many words in this new description will be left out. We
    can then use our weights for best predicting words, $\bm{\hat{x}}$ to
    estimate the salary of the job for this new description, now contained in
    the matrix $\bm{b}$. Our estimated salary will be stored in $\bm{\hat{b}}$,
    which we estimate from Equation~\ref{eq:linsolve}. \\

    The estimated salary is \$30,328.00. About \$15,000 different from the true
    salary associated with the job. How can we improve our analysis?

\end{document}




