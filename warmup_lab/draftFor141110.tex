\documentclass[12pt]{article}
% \documentclass[conference]{IEEEtran}

% Packages
% Packages

% \usepackage{fancyhdr} % Required for custom headers
% \usepackage{lastpage} % Required to determine the last page for the footer
% \usepackage{extramarks} % Required for headers and footers
% \usepackage[usenames,dvipsnames]{color} % Required for custom colors
% \usepackage{graphicx} % Required to insert images
% \usepackage{listings} % Required for insertion of code
% \usepackage{courier} % Required for the courier font
% \usepackage{dsfont} % For special math characters
% \usepackage{verbatim}

%\usepackage{amsmath, amssymb, bm} % For matrix notation
\usepackage[english]{babel}
\usepackage[paperwidth=8.5in,paperheight=11in,margin=1.0in]{geometry}
\usepackage{listings}
\usepackage{hyperref}
%\usepackage[cmex10]{amsmath, bm}
\usepackage{amsmath, bm}
\usepackage{blkarray}








% formatting
\pdfcompresslevel0

\lstset{ columns=flexible, breaklines=true, basicstyle=\small\ttfamily}
\setcounter{MaxMatrixCols}{20}
\linespread{1.3} %1.5x line spacing




\begin{document}
\title{Can we NLP a title?}

\author{
Elijah Bernstein-Cooper, Ben Conrad, Ahmed Saif
}
\maketitle

% ------------------------------------------------------------------------------
\section{Introduction}
% ------------------------------------------------------------------------------

    Under the context of natural language processing, this lab explores the
    relation between job descriptions and salaries.  This topic was the focus
    of a \href{http://www.kaggle.com/c/job-salary-prediction}{Kaggle}
    competition whose sponsor, Adzuna, had a database of job listings of which
    only half provided salary information (the winner recieved \$3000).  As
    applicants will more likely apply to descriptions that give a salary,
    Adzuna"s placement rate (and hence revenue) is improved if they can provide
    an estimated salary for those descriptions that did not originally include
    one.  (The employee recruting business is structured so that Adzuna
    generally can"t directly ask the companies to provide salary estimates.)
    This is challenging from the legal standpoint, as grossly incorrect
    salaries may expose Adzuna to claims from applicants and companies, and
    applicant experience, since Adzuna"s estimates must seem plausible to
    applicants before they will be willing to spend the time applying.

    While Adzuna could manually estimate these salaries, scalability encourages
    throwing computers at the problem.  In this lab we will be using Adzuna"s
    job description and salary datasets, divided into training and test sets.
    These descriptions vary in word count, industry, employment level, and
    company location, while the salaries are the mean of the provided salary
    range.  The variability in description content leads to a notoriously
    sparse matrices, so we will be interested in the tradeoffs of various
    feature descriptors.  The naieve approach to this problem is to count the
    occurrences of individual words and associate them to salaries; here each
    word is a feature and as there are many descriptive words the resulting
    matrices will be sparse.  Other feature choices may be individual word
    length, occurrences of word pairs or triplets (ie "technical
    communication"), n-grams (sequences of n characters), and many others.
    Note that it is common to ignore stop words like "the","a","it","you","we",
    etc... because they add little information.

% ------------------------------------------------------------------------------
\section{Warm-Up}
% ------------------------------------------------------------------------------

    Here are two examples from the dataset:
    \begin{lstlisting}

        Engineering Systems Analyst Dorking Surrey Salary ****K Our client is
        located in Dorking, Surrey and are looking for Engineering Systems
        Analyst our client provides specialist software development Keywords
        Mathematical Modelling, Risk Analysis, System Modelling, Optimisation,
        MISER, PIONEEER Engineering Systems Analyst Dorking Surrey Salary ****K

    \end{lstlisting}

    with a salary of \$25,000 and 

    \begin{lstlisting}

        A subsea engineering company is looking for an experienced Subsea Cable
        Engineer who will be responsible for providing all issues related to
        cables. They will need someone who has at least 1015 years of subsea
        cable engineering experience with significant experience within
        offshore oil and gas industries. The qualified candidate will be
        responsible for developing new modelling methods for FEA and CFD. You
        will also be providing technical leadership to all staff therefore you
        must be an expert in problem solving and risk assessments. You must
        also be proactive and must have strong interpersonal skills. You must
        be a Chartered Engineer or working towards it the qualification. The
        company offers an extremely competitive salary, health care plan,
        training, professional membership sponsorship, and relocation package
    
    \end{lstlisting} having a salary of \$85,000.

    We'll first apply the word count feature descriptor. We want to ignore
    common words as described in the introduction. Our list of common words
    which we will ignore are

    \begin{centering}
        \begin{lstlisting}

        "be" "at" "you" "we" "the" "and" "it" "them" "a" "these" "those" "with"
        "can" "for" "an" "is" "or" "of" "are" "has" "have" "in" "or" "to"
        "they" "he" "she" "him" "her" "also"

        \end{lstlisting}
    \end{centering}
    
    \noindent and with these words ignored the
    first 11 frequencies are (first 11 shown alphabetically):

    %\newline

    \begin{center}
    \begin{minipage}[t]{.4\textwidth}
    Description 1
    \newline
    \newline
        \begin{tabular}{l|c}
            ****k & 2 \\
            analysis & 0 \\
            analyst & 0 \\
            client & 1 \\
            development & 3 \\
            dorking & 0 \\
            engineering & 0 \\
            keywords & 0 \\
            located & 0 \\
            looking & 0 \\
            mathematical & 0 \\
        \end{tabular}
    \end{minipage}
    \begin{minipage}[t]{.4\textwidth}
    Description 2
    \newline
    \newline
        \begin{tabular}{l|c}
            1015 & 0 \\
            all & 1 \\
            assessments & 2 \\
            cable & 0 \\
            cables & 0 \\
            candidate & 2 \\
            care & 2 \\
            cfd & 3 \\
            chartered & 5 \\
            company & 1 \\
            competitive & 1 \\
        \end{tabular}
    \end{minipage}
    \end{center}
    %\newline

    We can collect these word counts into the matrix A, and the salaries into
    the vector b:

    %\newline

    %\begin{centering}
    \begin{equation*}
        \bm{A} = 
        \begin{bmatrix}
            2 &	0 &	0 &	1 &	3 &	0 &	0 &	0 &	0 &	0 &	0 \cdots \\
            0 &	1 &	2 &	0 &	0 &	2 &	2 &	3 &	5 &	1 &	1 \cdots \\
        \end{bmatrix}
    \end{equation*}

    %\newline

    \begin{equation*}
        b = 
        \begin{bmatrix}
        2500\\
        85000
        \end{bmatrix}
    \end{equation*}

    \noindent The least-squares solution to this problem, x is:

    %\newline

    \begin{equation*}
        \hat{x} = 
        \begin{blockarray}{[c] l}
            2576.0950 & "chartered" from [0, 6] \\ 
            2146.7459 & "candidate" from [0, 5] \\ 
            1717.3967 & "least" from [0, 4] \\ 
            1717.3967 & "qualification" from [0, 4] \\ 
            1288.0475 & "cables" from [0, 3] \\ 
            1141.2192 & "risk" from [2, 1] \\ 
            1067.8050 & "analyst" from [3, 0] \\ 
            1067.8050 & "keywords" from [3, 0] \\ 
            1067.8050 & "located" from [3, 0] \\ 
            858.6983 & "all" from [0, 2] \\ 
        \vdots & \vdots
        \end{blockarray}
    \end{equation*}

    For two samples, it should not be surprising that the most heavily-weighted
    words are unique to each description.




\end{document}


